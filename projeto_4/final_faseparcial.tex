\documentclass[11pt]{article}

% Fonte em português brasileiro
\usepackage[brazilian]{babel}
\usepackage[utf8]{inputenc}
\usepackage[T1]{fontenc}
\usepackage{lmodern}
\usepackage{enumitem}

% Identação e Margens
\usepackage{fullpage}      % Margens
\usepackage{indentfirst}   % Autoidentar

% Figuras
\usepackage{graphicx}       % Pictures
\graphicspath{{./figuras/}} % Path

\begin{document}
% Cabeçalho simples
\begin{center}\textbf{Universidade Federal de São Carlos}\end{center}
\begin{center}\textbf{Laboratório de Sistemas Operacionais}\end{center}

\noindent Gabriel de Paula \hfill 587710

\noindent Giovana Morais \hfill 587591

\noindent Mateus Abreu \hfill 612618

\noindent Thiago Borges da Silva \hfill 613770

\noindent \textbf{Projeto Final - Sitema de Arquivos - Fase intermediária}

\noindent \rule{\linewidth}{1.5pt}

% Início do texto
\section{Introdução}
Este relatório tem como objetivo entender mais a fundo a construção de software básico, construindo um sistema de arquivos elementar a partir de um dispositivo de disco genérico. Nele são apresentadas as soluções para resolver as funções da fase parcial do projeto final da Disciplina de Laboratório de Sistemas Operacionais.

\section{Função de Formatação - \textit{fs\_format()}}
\par
Inicialmente, definimos as 32 primeiras posições do vetor \textit{FAT} como agrupamento de \textit{FAT}, isso é, atribuir a estas posições o valor 3 (definido por uma constante). A próxima posição, ou seja, a posição 33 do vetor \textit{FAT}, definimos como agrupamento de diretórios, também definida por uma constante de valor 4. Todo o restante deste vetor é definido como agrupamento livre, que também foi definido por uma constante, que por sua vez possuía valor 2.
\par
Esta função também é responsável pela inicialização da struct de diretórios, atribuindo aos  atributos de cada posição deste vetor valores padrão (\textit{.size} = 0, \textit{.used} = 0).
\par
Por fim, esta função grava a formatação manipulada em memória na imagem, através de dois laços de repetição. O primeiro, é responsável por gravar as alterações até o inicio dos diretórios e o próximo é responsável por gravar do inicio dos diretórios até seu fim.

\section{Função de Inicialização -  \textit{fs\_init()}}
\par
Está função é responsável por carregar em memória os dados da imagem até o inicio do diretório e verificar se ele está formatado, se o disco estiver formatado, o próximo estágio é carregar o restante da imagem em memória, caso contrário, o sistema informa que o disco não está formatado.
\par
A verificação para saber se está formatado ou não é feita analisando apenas o início da \textit{FAT}, se este início seguir o padrão de 32 posições contendo agrupamento de \textit{FAT} e uma única posição contendo agrupamento de diretório, é considerado que está formatado.

\section{Funções de Espaços Livres e Listagem -  \textit{fs\_free()} e  \textit{fs\_list(char *buffer, int size)}}
\par
A tarefa desta função é retornar a quantidade de memória livre e isto é realizado através de uma verificação sequencial no vetor da \textit{FAT}. Por fim, este valor obtido é multipicado por uma constante que representa o tamanho de setor, ou seja, 512.
\par
função de listagem nesse paragrafo aquiiiiiiiii

\section{Função de Criação de Arquivo - \textit{fs\_create(char* file\_name)}}
\par
Inicialmente, é necessário encontrar os espaços livres do vetor da \textit{FAT} e no vetor de diretórios, caso não encontre, o sistema informa que não existe espaço livre, caso contrário, a posição livre encontrada anteriormente será atribuida como usada, o nome é escolhido a partir do nome que foi fornecido na chamada do \textit{fs\_create()}, seu tamanho será 0 e seu primeiro bloco é a primeira posição livre da \textit{FAT}, que também foi encontrada anteriormente.
\par
O paragráfo anteriormente descrevia a manipulação em memória, sendo assim, necessita-se de manipulação de disco, então novamente temos os dois laços citados na função de inicialização, que reescevem o \textit{buffer} inteiro. O primeiro laço reescreve até o inicio de diretórios e o segundo reescreve a parte de diretórios.

\section{Função de Remoção de Arquivo - \textit{fs\_remove(char* file\_name)}}
\par
É fornecido o nome do arquivo a ser removido ao chamar o \textit{fs\_remove()}, este nome é utilizado na busca sequencial e, se não encontrado, o sistema informa que o arquivo não existe naquele disco, caso contrário, a localização dos dados sofre um \textit{reset} (mesmo tratamento anteriormente feito no \textit{fs\_init()}). Novamente, é utilizado dos laços de repetição para a reescrita da imagem.


\section{Considerações Finais}
\par
A maior dificuldade encontrada foi a realização da aritmética, pois depois de realizar diversas operações, obtivemos um resultado que parecia estar correto, porém o sistema estava reescrevendo os agrupamentos de diretório como se fossem agrupamentos de FAT.
\par
Por fim, executamos o \textit{GHex} e confirmamos que a alteração nestas aritméticas concertou este \textit{bug} encontrado.


\end{document}
