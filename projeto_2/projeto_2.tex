\documentclass[12pt]{article}
\usepackage[utf8]{inputenc}
\usepackage[brazilian]{babel}

\author{Gabriel de Paula e Lima  587710\\
        Giovana Vieira de Morais  587591}
\title{Relatório Projeto 2}
\begin{document}

\maketitle

\newpage

\section*{Relatório}

As funcionalidades solicitadas para serem implementadas foram:

\begin{itemize}
\item{Permissão de recebimento de argumentos para os comandos executados}
\item{Suporte para execução de comandos em segundo plano}
\item{Redirecionamento de entrada e saída padrão para arquivos}
\end{itemize}

Explicaremos como implementamos cada uma dessas funções

\section*{Permissão de recebimento de argumentos para os comandos executados}
 A função supracitada foi implementada da seguinte maneira:

 Separamos cada string de entrada a fim de armazenarmos indivudualmente cada palavra digitada
 assim criando um vetor de argumentos 




% funções implementadas:
%   shell com funções com mais de um argumento
%       execv
%   comandos em segundo plano
%       waitpid + argumento pra não esperar um processo terminar antes de
%       iniciar o outro
%   comandos usando entrada e saída padrão
%

% iniciar uma seção: \section{Nome da seção}
% \begin{itemize}
% \item{bklabkabka}
% \end{itemize}
% \texttt <- pra trechos de código OU usar a biblioteca pseudoalg ou algo assim sei lá se vira (tem a algorithm tbm)

\end{document}
