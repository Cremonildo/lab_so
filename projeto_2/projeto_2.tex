\documentclass[12pt]{article}
\usepackage[utf8]{inputenc}
\usepackage[brazilian]{babel}
\usepackage{enumitem}

\author{Gabriel de Paula e Lima  587710\\
        Giovana Vieira de Morais  587591}
\title{Relatório Projeto 2}
\begin{document}

\maketitle

\newpage

\section*{A atividade}

A partir de um programa inicial que simula um interpretador de linhas de
comandos, devíamos implementar algumas funcionalidades, sendo elas:

\begin{description}[labelindent=1cm]
    \item[Permissão de argumentos para os comandos
        executados]{Alguns comandos, por exemplo \texttt{ls}, têm parâmetros
        adicionais que podem ser opcionais ou não. Dessa forma, nosso algoritmo
        deveria ser capaz de reconhecer esses parâmetros e enviar para a função
        de execução do sistema}
    \item[Suporte para execução de comandos em segundo plano]{Ao usar o
        caracter \texttt{\&}, o usuário indica que um comando deve ser executado
        em segundo plano. O algoritmo deveria ser capaz de diferenciar este
        símbolo de um parâmetro normal de execução.}
    \item[Redirecionamento de entrada e saída padrão para arquivos]
\end{description}

A seguir explicaremos a implementação de cada uma dessas funções

\subsection*{Permissão de recebimento de argumentos para os comandos executados}

Para cada string de entrada, separamos comandos e seus respectivos argumentos
em um vetor. 
 Então criamos um novo processo utilizando a chamada de sistema \texttt{fork},
 que cria esse novo processo idêntico ao processo-pai, tendo apenas o id
 (\texttt{pid}) diferente, que é sempre que é igual a 0.
 Após isso utilizamos a chamada de sistema \texttt{execvp},
 que irá substituir o conteúdo do processo filho pelo comando que queremos executar.
 Assim \texttt{execvp} recebe o comando digitado na linha de comando e o vetor
 de argumentos obtido por nós e então executa o comando e seus respectivos
 argumentos.

\subsection*{Suporte para execução de comandos em segundo plano}
Ao identificarmos no vetor de argumentos o símbolo de \texttt{\&}, o removemos e sinalizamos um flag de execução de segundo plano.
Após criarmos o novo processo através da chamada de sistema \texttt{fork},
verificamos se a flag de execução em segundo plano está sinalizada e utilizamos a chamada de sistema \texttt{waitpid},
que, por padrão, suspende a execução do processo-pai,
nesse caso a própria shell, até uma mudança de estado do processo-filho.
Porém utilizamos a opção \texttt{WNOHANG} a fim do processo não esperar
a mudança de estado do processo-filho, assim executando o comando requerido em segundo plano.

\subsection*{Redirecionamento de entrada e saída padrão para arquivos}
Para o redirecionamento de entrada, verificamos se há símbolo correspondente a redirecionamento de entrada (\texttt{<})
na entrada da shell.

Ao identificarmos o símbolo, armazenamos o nome do arquivo de entrada que vem logo após o \texttt{<}
e sinalizamos a flag de redirecionamento de entrada.
Todo o processo descrito até o momento é feito de modo igual para o redirecionamento de saída e seu símbolo (\texttt{>}).
Então criamos o novo processo através da chamada de sistema \texttt{fork}
e verificamos as flags de redirecionamentos.

Se tiverem sido sinalizadas, usamos a função de entrada e saída
\texttt{freopen}, redirecionando então os fluxos padrões para o arquivo o qual o nome já teve o armazenamento citado anteriormente.
No caso de ser redirecionamento de entrada, redirecionamos o arquivo para a entrada padrão \texttt{stdin},
no caso de redirecionamento de saída, redirecionamos a saída padrão \texttt{stdout} para o arquivo.

A chamada do redirecionamento foi feita após a chamada do \texttt{fork},
pois se o fizéssemos antes dele, estariamos fazendo o redirecionamento tanto no
processo-pai (shell) quanto no processo-filho, o que causou problemas no que
diz respeito a abertura e fechamento dos arquivos que iam ser usados.

% funções implementadas:
%   shell com funções com mais de um argumento
%       execvp
%   comandos em segundo plano
%       waitpid + argumento pra não esperar um processo terminar antes de
%       iniciar o outro
%   comandos usando entrada e saída padrão
%

% iniciar uma seção: \section{Nome da seção}
% \begin{itemize}
% \item{bklabkabka}
% \end{itemize}
% \texttt <- pra trechos de código OU usar a biblioteca pseudoalg ou algo assim sei lá se vira (tem a algorithm tbm)

\end{document}
