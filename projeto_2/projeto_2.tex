\documentclass[12pt]{article}
\usepackage[utf8]{inputenc}
\usepackage[brazilian]{babel}
\usepackage{enumitem}

\author{Gabriel de Paula e Lima  587710\\
        Giovana Vieira de Morais  587591}
\title{Relatório Projeto 2}
\begin{document}

\maketitle

\newpage

\section*{Relatório}

A partir de um programa inicial que simula um shell, as funcionalidades solicitadas para serem implementadas foram:

\begin{description}[labelindent=1cm]
    \item[Permissão de argumentos para os comandos
        executados]{Alguns comandos, por exemplo \texttt{ls}, têm parâmetros
        adicionais que podem ser opcionais ou não. Dessa forma, nosso algoritmo
        deveria ser capaz de reconhecer esses parâmetros e enviar para a função
        de execução do sistema}
    \item[Suporte para execução de comandos em segundo plano]{Ao usar o
        caracter \texttt{\&}, o usuário indica que um comando deve ser executado
        em segundo plano. O algoritmo deveria ser capaz de diferenciar este
        símbolo de um parâmetro normal de execução.}
    \item[Redirecionamento de entrada e saída padrão para arquivos]
\end{description}

A seguir explicaremos como implementamos cada uma dessas funções

\section*{Permissão de recebimento de argumentos para os comandos executados}
 A função supracitada foi implementada da seguinte maneira:

 Separamos cada string de entrada a fim de armazenarmos individualmente o comando e cada argumento digitado
 deste modo criando um vetor de argumentos.
 Então criamos um novo processo utilizando a chamada de sistema \texttt{fork},
 a qual cria esse novo processo idêntico ao processo-pai (Processo que chamou o \texttt{fork}),
 sendo o processo filho apenas diferente do pai em seu id (\texttt{pid}) que é igual a 0.
 Após isso utilizamos a chamada de sistema \texttt{execvp},
 a qual irá substituir o conteúdo do processo filho pelo comando que queremos executar.
 Assim \texttt{execvp} recebe o comando digitado na linha de comando e o vetor de argumentos obtido por nós.
 Para então executar o comando com seus argumentos.

\section*{Suporte para execução de comandos em segundo plano}
A função supracitada foi implementada da seguinte maneira:

Ao recebermos todos os argumentos do comando os armazenamos em um vetor de argumentos como citado no comando descrito anteriormente a este, verificamos se o símbolo de execução em segundo plano(\&) se encontra nesse vetor.
Se encontramos ele, o removemos e sinalizamos um flag de execução de segundo plano.
Após criarmos o novo processo através da chamada de sistema \texttt{fork},
verificamos se a flag de execução em segundo plano está sinalizada.
Estando ela sinalizada, utilizamos a chamada de sistema \texttt{waitpid},
chamada esta a qual por padrão normalmente suspende a execução do processo-pai,
nesse caso a própria shell, até uma mudança de estado do processo-filho.
Porém utilizamos a opção \texttt{WNOHANG} a fim do processo não esperar
a mudança de estado do processo-filho, assim executando o comando requerido em segundo plano.

\section{Redirecionamento de entrada e saída padrão para arquivos}
A função supracitada foi implementada da seguinte maneira:




% funções implementadas:
%   shell com funções com mais de um argumento
%       execvp
%   comandos em segundo plano
%       waitpid + argumento pra não esperar um processo terminar antes de
%       iniciar o outro
%   comandos usando entrada e saída padrão
%

% iniciar uma seção: \section{Nome da seção}
% \begin{itemize}
% \item{bklabkabka}
% \end{itemize}
% \texttt <- pra trechos de código OU usar a biblioteca pseudoalg ou algo assim sei lá se vira (tem a algorithm tbm)

\end{document}
