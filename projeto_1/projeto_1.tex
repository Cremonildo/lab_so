\documentclass[12pt]{article}
\usepackage[utf8]{inputenc}
\usepackage[brazilian]{babel}

\author{Gabriel de Paula e Lima  587710\\
        Giovana Vieira de Morais  587591}
\title{Relatório Projeto 1}
\begin{document}

\maketitle

O sistema total tem tamanho 4782492 KB, encontrado por meio do comando du
/home. 

A seguir, uma tabela que associa os valores às alterações feitas no arquivo
.config ao fazermos com que os módulos deixassem de ser módulos dinamicamente
carregados e, portanto, só usados quando necessário e passassem a ser estaticamente compilados, fazendo parte do núcleo
e sendo carregados toda vez que o sistema é bootado.

\begin{center}
    \begin{tabular}{ | l | l | l | }
        \hline
         & Núcleo Estático & Módulos \\
         \hline
        Sistema Original & 7210384 bytes & 59584 KB \\
        Sistema Alterado & 7519280 bytes & 164892 KB \\
        \hline
    \end{tabular}
\end{center}  

Ao fazer a instalação da nova parte estática, o núcleo compilado
é copiado para a pasta /boot. Devido a mudança do status de carregamento dos
módulos, que agora fazem parte do núcleo, há um aumento de tamanho.

É importante salientar que esse aumento de tamanho também implica maior
velocidade nos acessos às funcionalidades, uma vez que o acesso ao disco não ocorre com
frequência.

Num primeiro momento, estávamos mudando o status de carregamento dos módulos um
a um, procurando todas as dependências necessárias para tal. Contudo, notamos
que havia mais de três mil módulos, o que dificultaria o trabalho. Nesse caso,
alteramos todo arquivo .config de uma vez. 

\end{document}

